% Options for packages loaded elsewhere
\PassOptionsToPackage{unicode}{hyperref}
\PassOptionsToPackage{hyphens}{url}
\PassOptionsToPackage{dvipsnames,svgnames*,x11names*}{xcolor}
%
\documentclass[
  12pt,
  ignorenonframetext,
]{beamer}
\usepackage{pgfpages}
\setbeamertemplate{caption}[numbered]
\setbeamertemplate{caption label separator}{: }
\setbeamercolor{caption name}{fg=normal text.fg}
\beamertemplatenavigationsymbolsempty
% Prevent slide breaks in the middle of a paragraph
\widowpenalties 1 10000
\raggedbottom
\setbeamertemplate{part page}{
  \centering
  \begin{beamercolorbox}[sep=16pt,center]{part title}
    \usebeamerfont{part title}\insertpart\par
  \end{beamercolorbox}
}
\setbeamertemplate{section page}{
  \centering
  \begin{beamercolorbox}[sep=12pt,center]{part title}
    \usebeamerfont{section title}\insertsection\par
  \end{beamercolorbox}
}
\setbeamertemplate{subsection page}{
  \centering
  \begin{beamercolorbox}[sep=8pt,center]{part title}
    \usebeamerfont{subsection title}\insertsubsection\par
  \end{beamercolorbox}
}
\AtBeginPart{
  \frame{\partpage}
}
\AtBeginSection{
  \ifbibliography
  \else
    \frame{\sectionpage}
  \fi
}
\AtBeginSubsection{
  \frame{\subsectionpage}
}
\usepackage{lmodern}
\usepackage{amssymb,amsmath}
\usepackage{ifxetex,ifluatex}
\ifnum 0\ifxetex 1\fi\ifluatex 1\fi=0 % if pdftex
  \usepackage[T1]{fontenc}
  \usepackage[utf8]{inputenc}
  \usepackage{textcomp} % provide euro and other symbols
\else % if luatex or xetex
  \usepackage{unicode-math}
  \defaultfontfeatures{Scale=MatchLowercase}
  \defaultfontfeatures[\rmfamily]{Ligatures=TeX,Scale=1}
  \setmainfont[]{Roboto}
  \setsansfont[]{Noto Sans CJK JP}
  \setmonofont[]{Ricty Diminished}
  \ifxetex
    \usepackage{xeCJK}
    \setCJKmainfont[]{Noto Sans CJK JP}
  \fi
  \ifluatex
    \usepackage[]{luatexja-fontspec}
    \setmainjfont[]{Noto Sans CJK JP}
  \fi
\fi
\usefonttheme{serif} % use mainfont rather than sansfont for slide text
% Use upquote if available, for straight quotes in verbatim environments
\IfFileExists{upquote.sty}{\usepackage{upquote}}{}
\IfFileExists{microtype.sty}{% use microtype if available
  \usepackage[]{microtype}
  \UseMicrotypeSet[protrusion]{basicmath} % disable protrusion for tt fonts
}{}
\makeatletter
\@ifundefined{KOMAClassName}{% if non-KOMA class
  \IfFileExists{parskip.sty}{%
    \usepackage{parskip}
  }{% else
    \setlength{\parindent}{0pt}
    \setlength{\parskip}{6pt plus 2pt minus 1pt}}
}{% if KOMA class
  \KOMAoptions{parskip=half}}
\makeatother
\usepackage{xcolor}
\IfFileExists{xurl.sty}{\usepackage{xurl}}{} % add URL line breaks if available
\IfFileExists{bookmark.sty}{\usepackage{bookmark}}{\usepackage{hyperref}}
\hypersetup{
  pdftitle={R Markdownで日本語プレゼンテーション},
  pdfauthor={ill-identified},
  colorlinks=true,
  linkcolor=blue,
  filecolor=Maroon,
  citecolor=blue,
  urlcolor=magenta,
  pdfcreator={LaTeX via pandoc}}
\urlstyle{same} % disable monospaced font for URLs
\newif\ifbibliography
\usepackage{color}
\usepackage{fancyvrb}
\newcommand{\VerbBar}{|}
\newcommand{\VERB}{\Verb[commandchars=\\\{\}]}
\DefineVerbatimEnvironment{Highlighting}{Verbatim}{commandchars=\\\{\}}
% Add ',fontsize=\small' for more characters per line
\usepackage{framed}
\definecolor{shadecolor}{RGB}{248,248,248}
\newenvironment{Shaded}{\begin{snugshade}}{\end{snugshade}}
\newcommand{\AlertTok}[1]{\textcolor[rgb]{0.94,0.16,0.16}{#1}}
\newcommand{\AnnotationTok}[1]{\textcolor[rgb]{0.56,0.35,0.01}{\textbf{\textit{#1}}}}
\newcommand{\AttributeTok}[1]{\textcolor[rgb]{0.77,0.63,0.00}{#1}}
\newcommand{\BaseNTok}[1]{\textcolor[rgb]{0.00,0.00,0.81}{#1}}
\newcommand{\BuiltInTok}[1]{#1}
\newcommand{\CharTok}[1]{\textcolor[rgb]{0.31,0.60,0.02}{#1}}
\newcommand{\CommentTok}[1]{\textcolor[rgb]{0.56,0.35,0.01}{\textit{#1}}}
\newcommand{\CommentVarTok}[1]{\textcolor[rgb]{0.56,0.35,0.01}{\textbf{\textit{#1}}}}
\newcommand{\ConstantTok}[1]{\textcolor[rgb]{0.00,0.00,0.00}{#1}}
\newcommand{\ControlFlowTok}[1]{\textcolor[rgb]{0.13,0.29,0.53}{\textbf{#1}}}
\newcommand{\DataTypeTok}[1]{\textcolor[rgb]{0.13,0.29,0.53}{#1}}
\newcommand{\DecValTok}[1]{\textcolor[rgb]{0.00,0.00,0.81}{#1}}
\newcommand{\DocumentationTok}[1]{\textcolor[rgb]{0.56,0.35,0.01}{\textbf{\textit{#1}}}}
\newcommand{\ErrorTok}[1]{\textcolor[rgb]{0.64,0.00,0.00}{\textbf{#1}}}
\newcommand{\ExtensionTok}[1]{#1}
\newcommand{\FloatTok}[1]{\textcolor[rgb]{0.00,0.00,0.81}{#1}}
\newcommand{\FunctionTok}[1]{\textcolor[rgb]{0.00,0.00,0.00}{#1}}
\newcommand{\ImportTok}[1]{#1}
\newcommand{\InformationTok}[1]{\textcolor[rgb]{0.56,0.35,0.01}{\textbf{\textit{#1}}}}
\newcommand{\KeywordTok}[1]{\textcolor[rgb]{0.13,0.29,0.53}{\textbf{#1}}}
\newcommand{\NormalTok}[1]{#1}
\newcommand{\OperatorTok}[1]{\textcolor[rgb]{0.81,0.36,0.00}{\textbf{#1}}}
\newcommand{\OtherTok}[1]{\textcolor[rgb]{0.56,0.35,0.01}{#1}}
\newcommand{\PreprocessorTok}[1]{\textcolor[rgb]{0.56,0.35,0.01}{\textit{#1}}}
\newcommand{\RegionMarkerTok}[1]{#1}
\newcommand{\SpecialCharTok}[1]{\textcolor[rgb]{0.00,0.00,0.00}{#1}}
\newcommand{\SpecialStringTok}[1]{\textcolor[rgb]{0.31,0.60,0.02}{#1}}
\newcommand{\StringTok}[1]{\textcolor[rgb]{0.31,0.60,0.02}{#1}}
\newcommand{\VariableTok}[1]{\textcolor[rgb]{0.00,0.00,0.00}{#1}}
\newcommand{\VerbatimStringTok}[1]{\textcolor[rgb]{0.31,0.60,0.02}{#1}}
\newcommand{\WarningTok}[1]{\textcolor[rgb]{0.56,0.35,0.01}{\textbf{\textit{#1}}}}
\usepackage{longtable,booktabs}
\usepackage{caption}
% Make caption package work with longtable
\makeatletter
\def\fnum@table{\tablename~\thetable}
\makeatother
\usepackage{graphicx,grffile}
\makeatletter
\def\maxwidth{\ifdim\Gin@nat@width>\linewidth\linewidth\else\Gin@nat@width\fi}
\def\maxheight{\ifdim\Gin@nat@height>\textheight\textheight\else\Gin@nat@height\fi}
\makeatother
% Scale images if necessary, so that they will not overflow the page
% margins by default, and it is still possible to overwrite the defaults
% using explicit options in \includegraphics[width, height, ...]{}
\setkeys{Gin}{width=\maxwidth,height=\maxheight,keepaspectratio}
% Set default figure placement to htbp
\makeatletter
\def\fps@figure{htbp}
\makeatother
\usepackage[normalem]{ulem}
% Avoid problems with \sout in headers with hyperref
\pdfstringdefDisableCommands{\renewcommand{\sout}{}}
\setlength{\emergencystretch}{3em} % prevent overfull lines
\providecommand{\tightlist}{%
  \setlength{\itemsep}{0pt}\setlength{\parskip}{0pt}}
\setcounter{secnumdepth}{-\maxdimen} % remove section numbering
\usetheme[progressbar=frametitle,block=fill]{metropolis}
\makeatletter
\setlength{\metropolis@progressinheadfoot@linewidth}{2pt}
\usefonttheme{professionalfonts}
\usecolortheme{default}
\useinnertheme{default}
\useoutertheme{default}
\patchcmd{\beamer@sectionintoc}{\vskip1.5em}{\vskip0.5em}{}{}
\makeatother
\renewcommand{\figurename}{図}
\renewcommand{\tablename}{表}
\usepackage{bxtexlogo}
\colorlet{shadecolor}{gray!20}
\usepackage[numbers]{natbib}
\ifdefined\bibsection\renewcommand{\bibsection}{}\fi
\ifdefined\bibfont\renewcommand*{\bibfont}{\footnotesize}\fi
\usepackage{fmtcount}
\ifdefined\theFancyVerbLine\renewcommand{\theFancyVerbLine}{\small \padzeroes[2]{\decimal{FancyVerbLine}}}\fi
\IfFileExists{bxcoloremoji.sty}{\usepackage{bxcoloremoji}}{}
\usepackage[]{natbib}
\bibliographystyle{jecon}

\title{R Markdownで日本語\texttt{beamer}プレゼンテーション}
\author{ill-identified}
\date{2020-07-03}

\begin{document}
\frame{\titlepage}

\begin{frame}{目次}
\protect\hypertarget{ux76eeux6b21}{}

\tableofcontents[hideallsubsections]

\end{frame}

\hypertarget{ux30a4ux30f3ux30c8ux30edux30c0ux30afux30b7ux30e7ux30f3}{%
\section{イントロダクション}\label{ux30a4ux30f3ux30c8ux30edux30c0ux30afux30b7ux30e7ux30f3}}

\begin{frame}[fragile]{このスライドは何?}
\protect\hypertarget{ux3053ux306eux30b9ux30e9ux30a4ux30c9ux306fux4f55}{}

\begin{itemize}
\tightlist
\item
  あまり情報が流れていない,
  rmarkdownとbeamerで日本語を含むスライドを作るためのテンプレート兼用例集
\item
  \texttt{reveal.js}などhtml媒体は他の資料を参照

  \begin{itemize}
  \tightlist
  \item
    \href{https://kazutan.github.io/SappoRoR6/rmd_slide.html\#/}{ここ}や\href{https://kazutan.github.io/fukuokaR11/intro_rmarkdown_d.html}{ここ}を見よ
  \end{itemize}
\item
  もともとは自分用に作ったテンプレだったものを万人向けに修正
\end{itemize}

\end{frame}

\begin{frame}{想定される用途}
\protect\hypertarget{ux60f3ux5b9aux3055ux308cux308bux7528ux9014}{}

\begin{itemize}
\tightlist
\item
  Tokyo.R などRを使った話を発表する際の資料作成
\item
  技術・アカデミック寄りの話題を想定
\item
  具体的に要求されるもの

  \begin{itemize}
  \tightlist
  \item
    \textbf{日本語表示}
  \item
    ラスタまたはベクタ画像の挿入
  \item
    表の挿入
  \item
    Rコードを見やすく表示
  \item
    参考文献の相互参照/リスト自動生成
  \item
    \textbf{LyX やoverleafより簡単であること}
  \item
    \textbf{なんかナウでオサレな感じは求めてない}

    \begin{itemize}
    \tightlist
    \item
      自由すぎるデザインは不可
    \end{itemize}
  \end{itemize}
\end{itemize}

\end{frame}

\begin{frame}{先行事例の紹介}
\protect\hypertarget{ux5148ux884cux4e8bux4f8bux306eux7d39ux4ecb}{}

\begin{itemize}
\tightlist
\item
  伊東『\href{https://www.slideshare.net/hirokito/r-markdownbeamer-88777082}{R
  MarkdownとBeamerでプレゼンテーション資料作成}』

  \begin{itemize}
  \tightlist
  \item
    \LuaLaTeX を使って日本語でBeamerスライド作成する方法
  \end{itemize}
\item
  Atusy 『\href{https://blog.atusy.net/2019/05/14/rmd2pdf-any-font/}{R
  Markdown + XeLaTeX で日本語含め好きなフォントを使って PDF
  を出力する}』
\item
  先行事例との違い:

  \begin{itemize}
  \tightlist
  \item
    エンジンを\XeLaTeX に変更
  \item
    日本語文献bibファイル・bstファイルに対応
  \item
    スライド作例を多少充実させた
  \item
    その他体裁にこだわりたい人向け

    \begin{itemize}
    \tightlist
    \item
      「表X」「図X」といったキャプション
    \end{itemize}
  \end{itemize}
\end{itemize}

\end{frame}

\begin{frame}{\texttt{reveal.js} じゃダメなの?}
\protect\hypertarget{reveal.js-ux3058ux3083ux30c0ux30e1ux306aux306e}{}

\begin{itemize}
\tightlist
\item
  個人的にデザインとかあまり好きじゃない
\item
  上下左右に動いて空間識失調になる

  \begin{itemize}
  \tightlist
  \item
    (個人の体験です)
  \item
    上下のみにもできる
  \end{itemize}
\item
  htmlよりも不変な媒体にしたい

  \begin{itemize}
  \tightlist
  \item
    pdfが明確に優れているかは怪しい
  \end{itemize}
\item
  \sout{Q: お前が使いこなせてないだけじゃないの?}

  \begin{itemize}
  \tightlist
  \item
    \sout{A: うるさい}
  \end{itemize}
\end{itemize}

\end{frame}

\begin{frame}[fragile]{パワーポイントじゃダメなの?}
\protect\hypertarget{ux30d1ux30efux30fcux30ddux30a4ux30f3ux30c8ux3058ux3083ux30c0ux30e1ux306aux306e}{}

\begin{itemize}
\tightlist
\item
  私は\textbf{持ってない}
\item
  シンタックスハイライトが面倒

  \begin{itemize}
  \tightlist
  \item
    パワポの場合は\href{https://notchained.hatenablog.com/entry/2017/02/20/221446}{VSCode}か\href{https://reprex.tidyverse.org/articles/articles/rtf.html}{\texttt{reprex}}でコピペ
  \end{itemize}
\item
  ドラッグ\&ドロップで位置調整は便利
\item
  しかしポンチ絵芸術になりがち
\item
  極力シンプルにして視線誘導の負担をなくすべき

  \begin{itemize}
  \tightlist
  \item
    徹底するかは\textbf{好みの問題}
  \end{itemize}
\end{itemize}

\end{frame}

\begin{frame}[fragile]{技術的に厄介だったところ}
\protect\hypertarget{ux6280ux8853ux7684ux306bux5384ux4ecbux3060ux3063ux305fux3068ux3053ux308d}{}

\begin{itemize}
\tightlist
\item
  htmlとpdf(\LaTeX)とで微妙に違う挙動

  \begin{itemize}
  \tightlist
  \item
    ネット上の情報はhtml前提が多い
  \item
    pandocチョットワカル必要
  \end{itemize}
\item
  日本語を含む参考文献リスト

  \begin{itemize}
  \tightlist
  \item
    \upBibTeX の適用
  \item
    細かいオプション, 特に\texttt{metropolis}特有の仕様
  \end{itemize}
\item
  RStudio Cloud で動くかは未確認

  \begin{itemize}
  \tightlist
  \item
    日本語表示がおかしい説あり
  \end{itemize}
\end{itemize}

\end{frame}

\hypertarget{ux4f7fux3044ux65b9}{%
\section{使い方}\label{ux4f7fux3044ux65b9}}

\begin{frame}[fragile]{セットアップ}
\protect\hypertarget{ux30bbux30c3ux30c8ux30a2ux30c3ux30d7}{}

\begin{enumerate}
\tightlist
\item
  パッケージのインストール
\end{enumerate}

\begin{Shaded}
\begin{Highlighting}[]
\NormalTok{remotes}\OperatorTok{::}\KeywordTok{install_github}\NormalTok{(}
  \StringTok{"Gedevan-Aleksizde/my_latex_templates"}\NormalTok{,}
  \DataTypeTok{subdir =} \StringTok{"rmdCJK"}\NormalTok{)}
\end{Highlighting}
\end{Shaded}

\begin{enumerate}
\setcounter{enumi}{1}
\tightlist
\item
  TeXLive (\textgreater= 2018)のインストール
\end{enumerate}

\begin{itemize}
\tightlist
\item
  分からなければ\href{https://texwiki.texjp.org/?TeX\%20Live}{TeX wiki
  のページ}を参考に
\end{itemize}

\begin{enumerate}
\setcounter{enumi}{2}
\tightlist
\item
  (オプション)
  \href{https://github.com/matze/mtheme}{metropolisテーマ}のインストール
\end{enumerate}

\end{frame}

\begin{frame}[fragile]{基本}
\protect\hypertarget{ux57faux672c}{}

\begin{enumerate}
\tightlist
\item
  yamlヘッダに以下を書く
\end{enumerate}

\begin{Shaded}
\begin{Highlighting}[]
\FunctionTok{output}\KeywordTok{:}\AttributeTok{  rmdCKL::beamer_presentation_CJK}
\end{Highlighting}
\end{Shaded}

\begin{enumerate}
\setcounter{enumi}{1}
\tightlist
\item
  RStudioのツールバーの``Knit''を押す
\end{enumerate}

\begin{center}\includegraphics[width=0.8\linewidth]{/home/ks/R/x86_64-pc-linux-gnu-library/3.6/rmdCJK/extdata/img/render} \end{center}

\end{frame}

\begin{frame}[fragile]{フォント指定}
\protect\hypertarget{ux30d5ux30a9ux30f3ux30c8ux6307ux5b9a}{}

\begin{itemize}
\tightlist
\item
  使うマシンに応じて以下の箇所を適当に変える
\item
  初期設定はRictyを除き全て\href{https://fonts.google.com/?category=Sans+Serif\#standard-styles}{Google
  Fonts}で入手可
\item
  インラインでのフォント変更は\textbf{想定してない}
\end{itemize}

\begin{Shaded}
\begin{Highlighting}[]
\FunctionTok{mainfont}\KeywordTok{:}\AttributeTok{ Roboto}
\FunctionTok{sansfont}\KeywordTok{:}\AttributeTok{ Noto Sans CJK JP}
\FunctionTok{monofont}\KeywordTok{:}\AttributeTok{ Ricty Diminished}
\FunctionTok{CJKmainfont}\KeywordTok{:}\AttributeTok{ Noto Sans CJK JP}
\end{Highlighting}
\end{Shaded}

\end{frame}

\begin{frame}[fragile]{基本構文}
\protect\hypertarget{ux57faux672cux69cbux6587}{}

\begin{itemize}
\tightlist
\item
  markdown的な書き方でできる
\item
  ``\texttt{\#\#}タイトル'' でスライドの開始

  \begin{itemize}
  \tightlist
  \item
    \LaTeX コマンドも挿入可能
  \end{itemize}
\end{itemize}

\begin{Shaded}
\begin{Highlighting}[]
\FunctionTok{# 節見出し}
\FunctionTok{## タイトル1}
\NormalTok{- **太字** **bold**}
\NormalTok{- }\StringTok{_強調_ _emph_}
\StringTok{- }\BaseNTok{`タイプライタ体`}\StringTok{ }\BaseNTok{`mono`}
\end{Highlighting}
\end{Shaded}

\begin{itemize}
\tightlist
\item
  \textbf{太字} \textbf{bold}
\item
  \emph{強調} \emph{emph}
\item
  \texttt{タイプライタ体} \texttt{mono}
\end{itemize}

\end{frame}

\begin{frame}{BeamerやRMarkdown使用に役立つ資料}
\protect\hypertarget{beamerux3084rmarkdownux4f7fux7528ux306bux5f79ux7acbux3064ux8cc7ux6599}{}

\begin{itemize}
\tightlist
\item
  伊東『\href{https://www.slideshare.net/hirokito/r-markdownbeamer-88777082}{R
  MarkdownとBeamerでプレゼンテーション資料作成}』(\LuaLaTeX 使用)
\item
  松田『\href{http://ayapin-film.sakura.ne.jp/LaTeX/slides.html\#beamer}{Beamer読本-講演用スライド作成のために-}』
\item
  Kazutan『\href{https://kazutan.github.io/SappoRoR6/rmd_slide.html\#/}{R
  Markdownによるスライド生成}』『\href{https://kazutan.github.io/kazutanR/Rmd_intro.html}{R
  Markdown入門}』
\item
  Atusy『\href{https://blog.atusy.net/2019/05/14/rmd2pdf-any-font/}{R
  Markdown + XeLaTeX で日本語含め好きなフォントを使って PDF
  を出力する}』
\item
  R Markdown 2.0
  チートシートの\href{https://rstudio.com/wp-content/uploads/2016/11/Rmarkdown-cheatsheet-2.0_ja.pdf}{日本語訳},
  Takahashi, M.訳
\end{itemize}

\end{frame}

\begin{frame}{もう少しくわしいやつ}
\protect\hypertarget{ux3082ux3046ux5c11ux3057ux304fux308fux3057ux3044ux3084ux3064}{}

\begin{itemize}
\tightlist
\item
  Atusy 『\href{https://atusy.booth.pm/items/1453002}{R
  MarkdownユーザーのためのPandoc's Markdown}』
\item
  謝益輝 (yihui) ``\href{https://yihui.org/knitr/}{knitr - Elegant,
  flexible, and fast dynamic report generation with R}'' (開発者本人)
\item
  Xie, Yihui \& C. Dervieux
  ``\href{https://bookdown.org/yihui/rmarkdown-cookbook/}{R Markdown
  Coobook}''
\end{itemize}

\end{frame}

\begin{frame}[fragile]{今回使うパッケージ}
\protect\hypertarget{ux4ecaux56deux4f7fux3046ux30d1ux30c3ux30b1ux30fcux30b8}{}

\begin{itemize}
\tightlist
\item
  このファイル作成には以下を使用している

  \begin{itemize}
  \tightlist
  \item
    図表作成とか最低限必要なものだけ
  \end{itemize}
\end{itemize}

\begin{Shaded}
\begin{Highlighting}[numbers=left,,]
\KeywordTok{require}\NormalTok{(conflicted) }\CommentTok{# パッケージの競合防止用}
\KeywordTok{require}\NormalTok{(tidyverse)  }\CommentTok{# 全般}
\KeywordTok{require}\NormalTok{(ggthemes)   }\CommentTok{# ggplot2のデザイン変更}
\KeywordTok{require}\NormalTok{(ggdag)      }\CommentTok{# ネットワーク図の用例に}
\end{Highlighting}
\end{Shaded}

\begin{itemize}
\tightlist
\item
  以下はインストールのみ/読み込む必要なし

  \begin{itemize}
  \tightlist
  \item
    \texttt{citr}: 引用文献の挿入をGUIで
  \item
    \texttt{bookdown}: 数式をGUIで
  \end{itemize}
\end{itemize}

\end{frame}

\begin{frame}[fragile]{ソースコードの表示: 基本事項}
\protect\hypertarget{ux30bdux30fcux30b9ux30b3ux30fcux30c9ux306eux8868ux793a-ux57faux672cux4e8bux9805}{}

\begin{itemize}
\tightlist
\item
  \texttt{echo=T}でチャンク内コードを表示

  \begin{itemize}
  \tightlist
  \item
    デフォでは非表示
  \item
    \textbf{自動でシンタックスハイライト}
  \end{itemize}
\item
  はみ出す場合は\texttt{tidy=F}して手動改行

  \begin{itemize}
  \tightlist
  \item
    日本語等で折り返し地点がうまく行かない
  \end{itemize}
\item
  \texttt{class.source\ =\ "numberLines,\ LineAnchors"}
  で行番号表示(\href{https://blog.atusy.net/2019/04/18/rmd-line-num/}{参考})
\end{itemize}

\end{frame}

\begin{frame}[fragile]{ソースコードの表示: 出力例}
\protect\hypertarget{ux30bdux30fcux30b9ux30b3ux30fcux30c9ux306eux8868ux793a-ux51faux529bux4f8b}{}

\begin{verbatim}
```{r, echo=T, class.source = "numberLines, LineAnchors"}
require(conflicted) 
require(tidyverse)
require(ggthemes)
require(ggdag)
```
\end{verbatim}

\begin{Shaded}
\begin{Highlighting}[numbers=left,,]
\KeywordTok{require}\NormalTok{(conflicted)}
\KeywordTok{require}\NormalTok{(tidyverse)}
\KeywordTok{require}\NormalTok{(ggthemes)}
\KeywordTok{require}\NormalTok{(ggdag)}
\end{Highlighting}
\end{Shaded}

\end{frame}

\hypertarget{ux6570ux5f0fux95a2ux4fc2}{%
\section{数式関係}\label{ux6570ux5f0fux95a2ux4fc2}}

\begin{frame}[fragile]{数式の挿入: 行内(インライン)}
\protect\hypertarget{ux6570ux5f0fux306eux633fux5165-ux884cux5185ux30a4ux30f3ux30e9ux30a4ux30f3}{}

\begin{itemize}
\tightlist
\item
  markdown風のLaTeXコード埋め込み
\item
  \LaTeX の数式を\texttt{\$}で挟む
\item
  例: \texttt{らんま\$\textbackslash{}frac\{1\}\{2\}\$}

  \begin{itemize}
  \tightlist
  \item
    出力: らんま\(\frac{1}{2}\)
  \item
    注: 行内で分数はスラッシュ使ったほうが見やすい
  \end{itemize}
\item
  数式にはセリフフォント使用

  \begin{itemize}
  \tightlist
  \item
    スライドはサンセリフが良いとされる
  \item
    しかし数式の統一感がない
  \item
    (個人の好み?)
  \end{itemize}
\end{itemize}

\end{frame}

\begin{frame}[fragile]{数式の挿入: 独立行}
\protect\hypertarget{ux6570ux5f0fux306eux633fux5165-ux72ecux7acbux884c}{}

\begin{itemize}
\tightlist
\item
  \texttt{\$\$}で挟んだ範囲に\LaTeX 構文
\end{itemize}

\begin{Shaded}
\begin{Highlighting}[]
\SpecialStringTok{$$}\SpecialCharTok{\textbackslash{}begin}\SpecialStringTok{\{aligned\}}
\SpecialStringTok{& }\SpecialCharTok{\textbackslash{}sin}\SpecialStringTok{^2(x) + }\SpecialCharTok{\textbackslash{}cos}\SpecialStringTok{^2(x) = 1}\SpecialCharTok{\textbackslash{}\textbackslash{}}
\SpecialStringTok{& f(x) = }\SpecialCharTok{\textbackslash{}frac}\SpecialStringTok{\{1\}\{(2}\SpecialCharTok{\textbackslash{}pi}\SpecialStringTok{)^2\}}\SpecialCharTok{\textbackslash{}int}\SpecialStringTok{_\{}\SpecialCharTok{\textbackslash{}mathbb}\SpecialStringTok{\{R\}^n\}}
\SpecialCharTok{\textbackslash{}hat}\SpecialStringTok{\{f\}(}\SpecialCharTok{\textbackslash{}omega}\SpecialStringTok{)}\SpecialCharTok{\textbackslash{}exp}\SpecialStringTok{(i}\SpecialCharTok{\textbackslash{}omega}\SpecialStringTok{ x)d}\SpecialCharTok{\textbackslash{}omega}
\SpecialCharTok{\textbackslash{}end}\SpecialStringTok{\{aligned\}$$}
\end{Highlighting}
\end{Shaded}

\[\begin{aligned}
& \sin^2(x) + \cos^2(x) = 1\\
& f(x) = \frac{1}{(2\pi)^2}\int_{\mathbb{R}^n}\hat{f}(\omega)\exp(i\omega x)d\omega
\end{aligned}\]

\end{frame}

\begin{frame}[fragile]{数式の挿入: \texttt{bookdown}
パッケージのアドインで補完}
\protect\hypertarget{ux6570ux5f0fux306eux633fux5165-bookdown-ux30d1ux30c3ux30b1ux30fcux30b8ux306eux30a2ux30c9ux30a4ux30f3ux3067ux88dcux5b8c}{}

\begin{enumerate}
\tightlist
\item
  RStudioのツールバー ``Addins''
\item
  ``Input LaTeX Math''
\end{enumerate}

\begin{figure}

{\centering \includegraphics[width=1\linewidth,height=0.4\textheight]{/home/ks/R/x86_64-pc-linux-gnu-library/3.6/rmdCJK/extdata/img/math-input} 

}

\caption{bookdownの数式入力機能}\label{fig:math-input}
\end{figure}

\begin{itemize}
\tightlist
\item
  一部対応してない記号もある?

  \begin{itemize}
  \tightlist
  \item
    \texttt{\textbackslash{}mathbb\{\}}とか\texttt{\textbackslash{}hat\{\}}とか
  \end{itemize}
\item
  数式のみで\texttt{\textbackslash{}aligned}等環境の入力は不可
\end{itemize}

\end{frame}

\hypertarget{ux56f3ux8868ux306eux633fux5165}{%
\section{図表の挿入}\label{ux56f3ux8868ux306eux633fux5165}}

\begin{frame}[fragile]{図の挿入: 画像ファイル貼り付け}
\protect\hypertarget{ux56f3ux306eux633fux5165-ux753bux50cfux30d5ux30a1ux30a4ux30ebux8cbcux308aux4ed8ux3051}{}

\begin{itemize}
\tightlist
\item
  チャンクの\texttt{out.width=}/\texttt{out.height=}で調整
\item
  htmlと違い\textbf{アスペクト比は固定}
\item
  jpeg, png, eps, pdf に対応

  \begin{itemize}
  \tightlist
  \item
    gif, svg は上記いずれかに\textbf{手動で変換する}必要
  \item
    \LaTeX (\XeLaTeX) の制約
  \end{itemize}
\end{itemize}

\begin{Shaded}
\begin{Highlighting}[]
\NormalTok{knitr}\OperatorTok{::}\KeywordTok{include_graphics}\NormalTok{(}\KeywordTok{file.path}\NormalTok{(file_loc, }\KeywordTok{c}\NormalTok{(}\StringTok{"img/tiger.eps"}\NormalTok{, }\StringTok{"img/tiger.pdf"}\NormalTok{, }\StringTok{"img/tiger.png"}\NormalTok{)))}
\end{Highlighting}
\end{Shaded}

\begin{figure}

{\centering \includegraphics[width=0.2\linewidth]{/home/ks/R/x86_64-pc-linux-gnu-library/3.6/rmdCJK/extdata/img/tiger} \includegraphics[width=0.2\linewidth]{/home/ks/R/x86_64-pc-linux-gnu-library/3.6/rmdCJK/extdata/img/tiger} \includegraphics[width=0.2\linewidth]{/home/ks/R/x86_64-pc-linux-gnu-library/3.6/rmdCJK/extdata/img/tiger} 

}

\caption{いつもの虎(TeXLiveより)}\label{fig:unnamed-chunk-1}
\end{figure}

\end{frame}

\begin{frame}[fragile]{図の挿入: markdown構文で貼り付け}
\protect\hypertarget{ux56f3ux306eux633fux5165-markdownux69cbux6587ux3067ux8cbcux308aux4ed8ux3051}{}

\begin{itemize}
\tightlist
\item
  \texttt{out.width=}/\texttt{out.height=}が適用されない
\item
  pandoc構文でサイズ指定
\end{itemize}

\begin{Shaded}
\begin{Highlighting}[]
\AlertTok{![The Tiger](img/tiger.pdf)}\NormalTok{\{ height=30% \}}
\end{Highlighting}
\end{Shaded}

\begin{figure}
\centering
\includegraphics[width=\textwidth,height=0.3\textheight]{/home/ks/R/x86_64-pc-linux-gnu-library/3.6/rmdCJK/extdata/img/tiger.pdf}
\caption{The Tiger}
\end{figure}

\end{frame}

\begin{frame}[fragile]{図の挿入: \texttt{ggplot2}のグラフ}
\protect\hypertarget{ux56f3ux306eux633fux5165-ggplot2ux306eux30b0ux30e9ux30d5}{}

\begin{itemize}
\tightlist
\item
  \texttt{fig.cap=}でキャプションを設定可能.
  \texttt{labs(title\ =\ )}と違い自動相互参照あり
\end{itemize}

\begin{figure}

{\centering \includegraphics[width=1\linewidth,height=0.6\textheight]{beamer_linux_files/figure-beamer/plot-example-1} 

}

\caption{ggplot2の出力例: irisデータ}\label{fig:plot-example}
\end{figure}

\end{frame}

\begin{frame}[fragile]{図の挿入: 文字の大きさをそろえるには}
\protect\hypertarget{ux56f3ux306eux633fux5165-ux6587ux5b57ux306eux5927ux304dux3055ux3092ux305dux308dux3048ux308bux306bux306f}{}

\begin{itemize}
\tightlist
\item
  RStudioと出力された画像ファイルが違う!
\item
  グラフの文字小さすぎ!!
\item
  その原因は
\end{itemize}

\begin{enumerate}
\tightlist
\item
  \textbf{自動縮小される}ため
\end{enumerate}

\begin{itemize}
\tightlist
\item
  込み入った話なので\textbf{次のスライドへ}
\end{itemize}

\begin{enumerate}
\setcounter{enumi}{1}
\tightlist
\item
  \textbf{単位が違う}ため
\end{enumerate}

\begin{itemize}
\tightlist
\item
  beamerは主に \textbf{pt}単位
\item
  \texttt{ggplot2} は \texttt{aanotate()}のみ\textbf{mm}単位
\item
  補足

  \begin{itemize}
  \tightlist
  \item
    \texttt{cairo\_pdf()}の\texttt{pointsize}はビルトインデバイスにのみ影響
  \item
    『\href{https://uribo.hatenablog.com/entry/2018/06/11/232041}{ggplot2のsizeが意味するもの}』
  \end{itemize}
\end{itemize}

\end{frame}

\begin{frame}[fragile]{図の挿入: 画像サイズの基本ルール}
\protect\hypertarget{ux56f3ux306eux633fux5165-ux753bux50cfux30b5ux30a4ux30baux306eux57faux672cux30ebux30fcux30eb}{}

\begin{itemize}
\tightlist
\item
  Rが作図したファイルを一旦保存し, 拡大縮小して貼り付けられる

  \begin{itemize}
  \tightlist
  \item
    \texttt{fig.width}/\texttt{fig.height} は\textbf{保存時}のサイズ
  \item
    \texttt{out.width}/\texttt{out.height} は\textbf{表示する}サイズ
  \end{itemize}
\item
  Rの保存サイズとbeamerスライドのサイズのデフォルトは違う

  \begin{itemize}
  \tightlist
  \item
    スライドは\textbf{5.04 x 3.78 in (128 x 96 mm)}(4:3)
  \item
    \texttt{ggsave()}は \textbf{9.11 x 5.77 in} で保存
  \end{itemize}
\item
  RStudioのビューアは文字の大きさ\textbf{固定}で\textbf{サイズを画面に合わせる}

  \begin{itemize}
  \tightlist
  \item
    \textbf{違和感の正体}(?)
  \end{itemize}
\end{itemize}

\end{frame}

\begin{frame}[fragile]{図の挿入: 幅100\%で出力}
\protect\hypertarget{ux56f3ux306eux633fux5165-ux5e45100ux3067ux51faux529b}{}

\begin{itemize}
\tightlist
\item
  注:
  \texttt{out.width="100\%"}はスライドサイズではなく\textbf{本文領域の相対サイズ}
\end{itemize}

\begin{center}\includegraphics[width=1\linewidth]{beamer_linux_files/figure-beamer/plot-size-test1-1} \end{center}

\end{frame}

\begin{frame}{図の挿入: beamerサイズで保存, 幅100\%で出力}
\protect\hypertarget{ux56f3ux306eux633fux5165-beamerux30b5ux30a4ux30baux3067ux4fddux5b58-ux5e45100ux3067ux51faux529b}{}

\begin{itemize}
\tightlist
\item
  相対的に文字が大きくなった
\end{itemize}

\begin{center}\includegraphics[width=1\linewidth]{beamer_linux_files/figure-beamer/plot-size-test2-1} \end{center}

\end{frame}

\begin{frame}[fragile]{図の挿入: 字の大きさをなるべく揃える}
\protect\hypertarget{ux56f3ux306eux633fux5165-ux5b57ux306eux5927ux304dux3055ux3092ux306aux308bux3079ux304fux63c3ux3048ux308b}{}

\begin{itemize}
\tightlist
\item
  基準をbeamerに合わせる方法

  \begin{enumerate}
  \tightlist
  \item
    保存時サイズをbeamerの画面サイズと同じにする
  \end{enumerate}

  \begin{itemize}
  \tightlist
  \item
    このテンプレートのデフォルト設定
  \end{itemize}

  \begin{enumerate}
  \setcounter{enumi}{1}
  \tightlist
  \item
    \texttt{theme\_*()}で\texttt{base\_size}をbeamerの文字サイズと同じにする
  \end{enumerate}
\item
  out.width=``100\%''のとき, グラフタイトルと本文のサイズが一致
\item
  拡大縮小に合わせて文字の大きさを調整する
\item
  横長のグラフなら\texttt{fig.width=} を調整する
\end{itemize}

\end{frame}

\begin{frame}[fragile]{図の挿入: 再現可能なポンチ絵}
\protect\hypertarget{ux56f3ux306eux633fux5165-ux518dux73feux53efux80fdux306aux30ddux30f3ux30c1ux7d75}{}

\begin{itemize}
\tightlist
\item
  概念図とかの図示はどうするか

  \begin{itemize}
  \tightlist
  \item
    NOT データの視覚化(ビジュアライゼーション)
  \item
    \texttt{ggplot2}の本来の使い方ではない
  \end{itemize}
\item
  \texttt{ggdag} はネットワーク図に使える

  \begin{itemize}
  \tightlist
  \item
    因果ダイアグラム, 遷移図, グラフィカルモデル等
  \end{itemize}
\item
  \texttt{ggforce}
  は\href{https://rpubs.com/sdutky/559050}{ベン図の描画に応用可能}

  \begin{itemize}
  \tightlist
  \item
    世間的にはグラフの部分拡大用パッケージ?
  \end{itemize}
\item
  詳しくは個別のマニュアル参照
\item
  霞が関流ポンチ絵は\textbf{専門外}
\end{itemize}

\end{frame}

\begin{frame}{図の挿入: ポンチ絵の例1}
\protect\hypertarget{ux56f3ux306eux633fux5165-ux30ddux30f3ux30c1ux7d75ux306eux4f8b1}{}

\begin{itemize}
\tightlist
\item
  \href{https://speakerdeck.com/ktgrstsh/r-and-epidemical-mathematical-models}{以前作ったやつ}の修正
\end{itemize}

\begin{figure}

{\centering \includegraphics[width=0.9\linewidth,height=0.7\textheight]{beamer_linux_files/figure-beamer/punch-chart-example-1} 

}

\caption{ggdagで作ったYJ-SEIRモデルの遷移図}\label{fig:punch-chart-example}
\end{figure}

\end{frame}

\begin{frame}[fragile]{図の挿入: ポンチ絵の例2}
\protect\hypertarget{ux56f3ux306eux633fux5165-ux30ddux30f3ux30c1ux7d75ux306eux4f8b2}{}

\begin{itemize}
\tightlist
\item
  \texttt{ggforce::geom\_circle()} を利用

  \begin{itemize}
  \tightlist
  \item
    参考:
    \href{https://scriptsandstatistics.wordpress.com/2018/04/26/how-to-plot-venn-diagrams-using-r-ggplot2-and-ggforce/}{How
    to Plot Venn Diagrams Using R, ggplot2 and ggforce}
  \end{itemize}
\end{itemize}

\begin{figure}

{\centering \includegraphics[width=1\linewidth,height=0.6\textheight]{beamer_linux_files/figure-beamer/venn-1} 

}

\caption{ベン図の例}\label{fig:venn}
\end{figure}

\end{frame}

\begin{frame}[fragile]{図の挿入: R以外のデバイス}
\protect\hypertarget{ux56f3ux306eux633fux5165-rux4ee5ux5916ux306eux30c7ux30d0ux30a4ux30b9}{}

\begin{itemize}
\tightlist
\item
  \LaTeX の\texttt{tikz}を使用可能

  \begin{itemize}
  \tightlist
  \item
    \texttt{tikz}を知らない人は\href{https://www.opt.mist.i.u-tokyo.ac.jp/~tasuku/tikz.html}{ここ}や\href{https://texwiki.texjp.org/?TikZ}{TeX
    Wiki}を読む
  \item
    現時点では\textbf{日本語表示が面倒}
    (\href{https://stackoverflow.com/questions/51689570/how-to-force-tikz-in-rmarkdown-document-to-show-cyrillic-text}{参考})
  \item
    \sout{そこまでやるなら全部\LaTeX で書いたほうがいいのではないか?}
  \end{itemize}
\item
  dot言語単体で実行することも可能
\end{itemize}

\end{frame}

\begin{frame}[fragile]{表の挿入: データフレーム}
\protect\hypertarget{ux8868ux306eux633fux5165-ux30c7ux30fcux30bfux30d5ux30ecux30fcux30e0}{}

\begin{itemize}
\tightlist
\item
  Rのデータフレームとして作成して出す

  \begin{itemize}
  \tightlist
  \item
    はみ出す場合は縮小
  \item
    最低限の情報だけ掲載するのは大前提
  \item
    \texttt{df\_print:\ kable}
    では\texttt{caption}指定が\href{https://stackoverflow.com/questions/48410861/how-to-add-table-caption-using-df-print}{ややこしい}
  \end{itemize}
\end{itemize}

\begin{Shaded}
\begin{Highlighting}[]
\KeywordTok{data}\NormalTok{(iris)}
\NormalTok{knitr}\OperatorTok{::}\KeywordTok{kable}\NormalTok{(}\KeywordTok{head}\NormalTok{(iris[, }\DecValTok{1}\OperatorTok{:}\DecValTok{3}\NormalTok{]),}
             \DataTypeTok{caption=}\StringTok{"kable()による表示"}\NormalTok{)}
\end{Highlighting}
\end{Shaded}

\end{frame}

\begin{frame}[fragile]{表の挿入: データフレームを\texttt{kable()}で表示}
\protect\hypertarget{ux8868ux306eux633fux5165-ux30c7ux30fcux30bfux30d5ux30ecux30fcux30e0ux3092kableux3067ux8868ux793a}{}

\begin{Shaded}
\begin{Highlighting}[]
\KeywordTok{data}\NormalTok{(iris)}
\NormalTok{knitr}\OperatorTok{::}\KeywordTok{kable}\NormalTok{(}\KeywordTok{head}\NormalTok{(iris[, }\DecValTok{1}\OperatorTok{:}\DecValTok{3}\NormalTok{]),}
             \DataTypeTok{caption=}\StringTok{"kable()による表示"}\NormalTok{)}
\end{Highlighting}
\end{Shaded}

\begin{longtable}[]{@{}rrr@{}}
\caption{kable()による表示}\tabularnewline
\toprule
Sepal.Length & Sepal.Width & Petal.Length\tabularnewline
\midrule
\endfirsthead
\toprule
Sepal.Length & Sepal.Width & Petal.Length\tabularnewline
\midrule
\endhead
5.1 & 3.5 & 1.4\tabularnewline
4.9 & 3.0 & 1.4\tabularnewline
4.7 & 3.2 & 1.3\tabularnewline
4.6 & 3.1 & 1.5\tabularnewline
5.0 & 3.6 & 1.4\tabularnewline
5.4 & 3.9 & 1.7\tabularnewline
\bottomrule
\end{longtable}

\end{frame}

\begin{frame}[fragile]{表の挿入: \LaTeX コード}
\protect\hypertarget{ux8868ux306eux633fux5165-ux30b3ux30fcux30c9}{}

\begin{itemize}
\tightlist
\item
  \LaTeX のコードを貼り付けて表を掲載

  \begin{itemize}
  \tightlist
  \item
    \texttt{\textbackslash{}input\{tab.tex\}} でコピペなしで貼り付け可
  \item
    \texttt{stargazer}との併用
  \item
    \textbf{リサイズは手動で}
  \end{itemize}
\item
  以下, 表を\texttt{.tex}で出力してから読み込む
\end{itemize}

\begin{Shaded}
\begin{Highlighting}[]
\NormalTok{xtable}\OperatorTok{::}\KeywordTok{xtable}\NormalTok{(}
  \KeywordTok{head}\NormalTok{(iris), }\DataTypeTok{caption =} \StringTok{"xtableでexport"}\NormalTok{) }\OperatorTok
\StringTok{  }\KeywordTok{print}\NormalTok{(}\DataTypeTok{file =} \StringTok{"tab.tex"}\NormalTok{)}
\end{Highlighting}
\end{Shaded}

\input{tab.tex}

\end{frame}

\begin{frame}[fragile]{表の挿入: markdown構文}
\protect\hypertarget{ux8868ux306eux633fux5165-markdownux69cbux6587}{}

\small

\begin{verbatim}
Table: 得点一覧

  クラス 科目   平均
  ------ ----- -----
  A      算数   $90$
  B      算数   $95$
  ------ ----- -----
\end{verbatim}

\normalsize

\begin{longtable}[]{@{}llc@{}}
\caption{得点一覧}\tabularnewline
\toprule
クラス & 科目 & 平均\tabularnewline
\midrule
\endfirsthead
\toprule
クラス & 科目 & 平均\tabularnewline
\midrule
\endhead
A & 算数 & \(90\)\tabularnewline
B & 算数 & \(95\)\tabularnewline
\bottomrule
\end{longtable}

\end{frame}

\hypertarget{ux5916ux90e8ux8cc7ux6599ux306eux5f15ux7528ux65b9ux6cd5}{%
\section{外部資料の引用方法}\label{ux5916ux90e8ux8cc7ux6599ux306eux5f15ux7528ux65b9ux6cd5}}

\begin{frame}[fragile]{ハイパーリンクの挿入}
\protect\hypertarget{ux30cfux30a4ux30d1ux30fcux30eaux30f3ux30afux306eux633fux5165}{}

\begin{itemize}
\tightlist
\item
  urlは自動でリンク

  \begin{itemize}
  \tightlist
  \item
    \url{https://rstudio.com/}
  \end{itemize}
\item
  markdown方式のリンク

  \begin{itemize}
  \tightlist
  \item
    \texttt{{[}RStudio{]}(https://rstudio.com/)}
  \item
    \href{https://rstudio.com/}{RStudio}
  \end{itemize}
\item
  画像にハイパーリンク
  \href{https://rstudio.com/}{\includegraphics[width=\textwidth,height=0.1\textheight]{/home/ks/R/x86_64-pc-linux-gnu-library/3.6/rmdCJK/extdata/img/RStudio-Logo-flat.pdf}}
  を貼ることも可
\end{itemize}

\end{frame}

\begin{frame}[fragile]{文献引用の方法}
\protect\hypertarget{ux6587ux732eux5f15ux7528ux306eux65b9ux6cd5}{}

\begin{itemize}
\tightlist
\item
  \texttt{{[}@ref{]}} で番号引用: \texttt{\textbackslash{}citep\{ref\}}
  に対応 (\texttt{{[}1{]}})
\item
  \texttt{@ref} で著者名引用:
  \texttt{\textbackslash{}citet\{ref\}}に対応
  (\texttt{hogehoge\ et\ al.})
\item
  \texttt{{[}@ref1;\ @ref1{]}} で連番引用 \texttt{{[}1,\ 2{]}}
\item
  以下引用テスト
\end{itemize}

\begin{Shaded}
\begin{Highlighting}[]
\NormalTok{[@R-base; @R-bookdown; @R-citr; @wickham2016Data]}
\end{Highlighting}
\end{Shaded}

\citep{R-base, R-bookdown, R-citr, wickham2016Data}

\end{frame}

\begin{frame}[fragile]{文献引用の補助: 引用子の補完}
\protect\hypertarget{ux6587ux732eux5f15ux7528ux306eux88dcux52a9-ux5f15ux7528ux5b50ux306eux88dcux5b8c}{}

\begin{itemize}
\tightlist
\item
  重複・書き間違えの防止
\item
  \texttt{citr}パッケージを使うと楽

  \begin{itemize}
  \tightlist
  \item
    ツールバーの \texttt{Addins} から選択
  \item
    zotero連携機能あり
  \end{itemize}
\end{itemize}

\begin{figure}

{\centering \includegraphics[width=0.5\linewidth]{/home/ks/R/x86_64-pc-linux-gnu-library/3.6/rmdCJK/extdata/img/citr} 

}

\caption{citrパッケージのGUI}\label{fig:citr-image}
\end{figure}

\end{frame}

\begin{frame}[fragile]{文献引用の補助: 文献管理}
\protect\hypertarget{ux6587ux732eux5f15ux7528ux306eux88dcux52a9-ux6587ux732eux7ba1ux7406}{}

\begin{itemize}
\tightlist
\item
  Mendeley, Zotero, ReabCubeの3つが多い?
\item
  私はZoteroを使っている

  \begin{itemize}
  \tightlist
  \item
    多言語対応, 連携機能の充実, 料金などの理由
  \item
    参考:
    『\href{https://ill-identified.hatenablog.com/entry/2019/03/05/195257}{Mendeley
    Exodus Mendeley から Zotero への移行の手引き\textasciitilde{}}』
  \end{itemize}
\item
  \texttt{RefManageR} パッケージ

  \begin{itemize}
  \tightlist
  \item
    Rでbibファイルをパースしたりする
  \item
    文献管理用には既存ソフトで十分?
  \end{itemize}
\end{itemize}

\end{frame}

\hypertarget{ux305dux306eux4ed6ux306eux6a5fux80fd}{%
\section{その他の機能}\label{ux305dux306eux4ed6ux306eux6a5fux80fd}}

\begin{frame}[fragile]{絵文字}
\protect\hypertarget{ux7d75ux6587ux5b57}{}

\begin{itemize}
\tightlist
\item
  \href{https://github.com/zr-tex8r/BXcoloremoji}{\texttt{BXcoloremoji}}をインストールすれば可能

  \begin{itemize}
  \tightlist
  \item
    \texttt{\textbackslash{}coloremoji\{\}} で絵文字表示:
    \ifdefined\coloremoji \coloremoji{🍣} \else (ここに絵文字) \fi
  \end{itemize}
\item
  グラフ描画には特に設定必要なし

  \begin{itemize}
  \tightlist
  \item
    ソースコード上のものは文字化けする
  \end{itemize}
\end{itemize}

\begin{Shaded}
\begin{Highlighting}[]
\KeywordTok{plot}\NormalTok{(}\DecValTok{1}\OperatorTok{:}\DecValTok{10}\NormalTok{, }\DataTypeTok{pch =} \StringTok{"🍣"}\NormalTok{)}
\end{Highlighting}
\end{Shaded}

\begin{center}\includegraphics[width=0.4\linewidth]{beamer_linux_files/figure-beamer/sushi-plot-1} \end{center}

\end{frame}

\hypertarget{ux57faux672cux7684ux306aux30abux30b9ux30bfux30deux30a4ux30ba}{%
\section{基本的なカスタマイズ}\label{ux57faux672cux7684ux306aux30abux30b9ux30bfux30deux30a4ux30ba}}

\begin{frame}[fragile]{フォント変更}
\protect\hypertarget{ux30d5ux30a9ux30f3ux30c8ux5909ux66f4}{}

\begin{itemize}
\tightlist
\item
  \texttt{XeCJK} パッケージで制御している
\end{itemize}

\begin{Shaded}
\begin{Highlighting}[]
\FunctionTok{mainfont}\KeywordTok{:}\AttributeTok{ Roboto}
\FunctionTok{sansfont}\KeywordTok{:}\AttributeTok{ Noto Sans CJK JP}
\FunctionTok{monofont}\KeywordTok{:}\AttributeTok{ Ricty Diminished}
\FunctionTok{CJKmainfont}\KeywordTok{:}\AttributeTok{ Noto Sans CJK JP}
\end{Highlighting}
\end{Shaded}

\end{frame}

\begin{frame}[fragile]{スライドのテーマ変更}
\protect\hypertarget{ux30b9ux30e9ux30a4ux30c9ux306eux30c6ux30fcux30deux5909ux66f4}{}

\begin{itemize}
\tightlist
\item
  指定できる名前一覧は\href{https://deic-web.uab.cat/~iblanes/beamer_gallery/index.html}{ここ}を参照

  \begin{itemize}
  \tightlist
  \item
    fonttheme のデフォルト値は \texttt{professionalfont}
  \item
    metropolis はあまり変化がない
  \end{itemize}
\end{itemize}

\begin{Shaded}
\begin{Highlighting}[]
\FunctionTok{output}\KeywordTok{:}
\AttributeTok{  rmdCJK:}\FunctionTok{:beamer_presentation_CJK}\KeywordTok{:}
\AttributeTok{    }\FunctionTok{theme}\KeywordTok{:}\AttributeTok{ metropolis}
\AttributeTok{    }\FunctionTok{colortheme}\KeywordTok{:}\AttributeTok{ yellow}
\AttributeTok{    }\FunctionTok{outertheme}\KeywordTok{:}\AttributeTok{ default}
\AttributeTok{    }\FunctionTok{innertheme}\KeywordTok{:}\AttributeTok{ default}
\AttributeTok{    }\FunctionTok{fonttheme}\KeywordTok{:}\AttributeTok{ default}
\end{Highlighting}
\end{Shaded}

\end{frame}

\begin{frame}[fragile]{シンタックスハイライトのテーマ変更}
\protect\hypertarget{ux30b7ux30f3ux30bfux30c3ux30afux30b9ux30cfux30a4ux30e9ux30a4ux30c8ux306eux30c6ux30fcux30deux5909ux66f4}{}

\begin{itemize}
\tightlist
\item
  テーマは以下が用意されている

  \begin{itemize}
  \tightlist
  \item
    \texttt{default}, \texttt{tango}, \texttt{pygments}, \texttt{kate},
    \texttt{monochrome}, \texttt{espresso}, \texttt{zenburn},
    \texttt{haddock}, \texttt{breezedark}, \texttt{textmate}

    \begin{itemize}
    \tightlist
    \item
      参考\href{https://bookdown.org/yihui/rmarkdown/html-document.html}{Xie
      Yihui のドキュメント}
    \end{itemize}
  \end{itemize}
\end{itemize}

\begin{Shaded}
\begin{Highlighting}[]
\FunctionTok{output}\KeywordTok{:}
\AttributeTok{  rmdCJK:}\FunctionTok{:beamer_presentation_CJK}\KeywordTok{:}
\AttributeTok{  }\FunctionTok{highlight}\KeywordTok{:}\AttributeTok{ tango}
\end{Highlighting}
\end{Shaded}

\end{frame}

\begin{frame}[fragile]{色の変更}
\protect\hypertarget{ux8272ux306eux5909ux66f4}{}

\begin{itemize}
\tightlist
\item
  ハイパーリンクの色を変えたい場合は以下をいじる

  \begin{itemize}
  \tightlist
  \item
    \texttt{linkcolor} スライド内リンク
  \item
    \texttt{citecolor} 参考文献リストへのリンク
  \item
    \texttt{urlcolor} urlリンク
  \end{itemize}
\item
  デフォルトで使用できる色名は\href{http://www.latex-cmd.com/style/color.html}{ここ}を参照
\end{itemize}

\begin{Shaded}
\begin{Highlighting}[]
\FunctionTok{output}\KeywordTok{:}
\AttributeTok{  rmdCJK:}\FunctionTok{:beamer_presentation_CJK}\KeywordTok{:}
\AttributeTok{    }\FunctionTok{linkcolor}\KeywordTok{:}\AttributeTok{ blue}
\AttributeTok{    }\FunctionTok{citecolor}\KeywordTok{:}\AttributeTok{ green}
\AttributeTok{    }\FunctionTok{urlcolor}\KeywordTok{:}\AttributeTok{ red}
\end{Highlighting}
\end{Shaded}

\end{frame}

\begin{frame}[fragile]{引用形式の変更}
\protect\hypertarget{ux5f15ux7528ux5f62ux5f0fux306eux5909ux66f4}{}

\begin{itemize}
\tightlist
\item
  デフォルトでは \texttt{natbib} パッケージを使用
\item
  デフォルトでは \texttt{{[}1{]}} のような番号形式
\item
  著者(年) 形式にしたい場合は \texttt{authoryear}

  \begin{itemize}
  \tightlist
  \item
    その他のオプションは\href{http://texdoc.net/texmf-dist/doc/latex/natbib/natnotes.pdf}{natnotes.pdf}を参照
  \end{itemize}
\item
  \texttt{biblatex}/\texttt{biber} の使用は\textbf{想定していない}
\end{itemize}

\begin{Shaded}
\begin{Highlighting}[]
\FunctionTok{output}\KeywordTok{:}
\AttributeTok{  rmdCJK:}\FunctionTok{:beamer_presentation_CJK}\KeywordTok{:}
\AttributeTok{    }\FunctionTok{citation-package}\KeywordTok{:}\AttributeTok{ natbib}
\AttributeTok{    }\FunctionTok{citation-options}\KeywordTok{:}\AttributeTok{ authoryear}
\end{Highlighting}
\end{Shaded}

\end{frame}

\begin{frame}[fragile]{参考文献リストの変更}
\protect\hypertarget{ux53c2ux8003ux6587ux732eux30eaux30b9ux30c8ux306eux5909ux66f4}{}

\begin{itemize}
\tightlist
\item
  \texttt{.bib}, \texttt{.bst} は以下にファイルパスを指定する
\item
  \texttt{.bst} は TeX側が認識していればフルパス・相対パスである必要なし
\end{itemize}

\begin{Shaded}
\begin{Highlighting}[]
\FunctionTok{bibliography}\KeywordTok{:}\AttributeTok{ examples.bib}
\FunctionTok{biblio-style}\KeywordTok{:}\AttributeTok{ jecon}
\end{Highlighting}
\end{Shaded}

\end{frame}

\begin{frame}[fragile]{「図」「表」の表示}
\protect\hypertarget{ux56f3ux8868ux306eux8868ux793a}{}

\begin{itemize}
\tightlist
\item
  図や表を掲載すると自動で「図X」「表Y」などと表示される

  \begin{itemize}
  \tightlist
  \item
    ``Fig.'', ``Tab.'' などと表示したい場合は以下のように変更
  \end{itemize}
\end{itemize}

\begin{Shaded}
\begin{Highlighting}[]
\FunctionTok{output}\KeywordTok{:}
\AttributeTok{  rmdCJK:}\FunctionTok{:beamer_presentation_CJK}\KeywordTok{:}
\AttributeTok{    }\FunctionTok{figurename}\KeywordTok{:}\AttributeTok{ Fig.}
\AttributeTok{    }\FunctionTok{tablename}\KeywordTok{:}\AttributeTok{ Tab.}
\end{Highlighting}
\end{Shaded}

\end{frame}

\hypertarget{ux30c8ux30e9ux30d6ux30ebux30b7ux30e5ux30fcux30c6ux30a3ux30f3ux30b0}{%
\section{トラブルシューティング}\label{ux30c8ux30e9ux30d6ux30ebux30b7ux30e5ux30fcux30c6ux30a3ux30f3ux30b0}}

\begin{frame}[fragile]{Q: エラーの原因がよくわからない}
\protect\hypertarget{q-ux30a8ux30e9ux30fcux306eux539fux56e0ux304cux3088ux304fux308fux304bux3089ux306aux3044}{}

\begin{itemize}
\tightlist
\item
  A: \textbf{キャッシュ削除すると良くなることもある}

  \begin{itemize}
  \tightlist
  \item
    (\textbf{叩けば直る}レベルの雑アドバイス)
  \item
    \texttt{\{ファイル名\}\_cache}, \texttt{\{ファイル名\}\_files}
    というディレクトリを消す
  \item
    前回失敗した際のキャッシュが悪さしてることは結構ある
  \item
    または \texttt{cache\ =\ F}, \texttt{keep\_tex:\ False},
    \texttt{keep\_md:\ False} でキャッシュを残さない
  \item
    エラーメッセージが実態と矛盾してるときはまず試す
  \end{itemize}
\item
  A: \texttt{rmarkdown}/\texttt{knitr}と\LaTeX どちらのエラーか確認

  \begin{itemize}
  \tightlist
  \item
    \texttt{output\ file:\ \{ファイル名\}.md}
    と出ればpandocまでは機能している
  \item
    pandocの変換が意図したものでない可能性はある
  \end{itemize}
\end{itemize}

\end{frame}

\hypertarget{ux307eux3068ux3081}{%
\section{まとめ}\label{ux307eux3068ux3081}}

\begin{frame}[fragile]{結果どうなったか}
\protect\hypertarget{ux7d50ux679cux3069ux3046ux306aux3063ux305fux304b}{}

\begin{itemize}
\tightlist
\item
  \textbf{良く}なったこと

  \begin{itemize}
  \tightlist
  \item
    \texttt{lstlisting.sty}\textbf{より見やすい}シンタックスハイライト
  \item
    Rの画像や数値出力を\textbf{コピペしなくて済む}
  \item
    一画面に収めるための構成だけ考えれば済むように
  \end{itemize}
\item
  \textbf{悪く}なったこと

  \begin{itemize}
  \tightlist
  \item
    (パワポユーザ的に)WYSIWYGでないので作りづらい?
  \item
    数式のリアルタイムレンダリング/補完はLyXが依然優秀
  \item
    python作業中(jupyter notebookへの)\textbf{不満高まり}
  \item
    ポンチ絵も\texttt{ggplot2}で作らねばという\textbf{強迫症状}
  \end{itemize}
\end{itemize}

\end{frame}

\begin{frame}[fragile]{改良・機能追加したいところ}
\protect\hypertarget{ux6539ux826fux6a5fux80fdux8ffdux52a0ux3057ux305fux3044ux3068ux3053ux308d}{}

\begin{itemize}
\tightlist
\item
  手動セットアップ作業の削減

  \begin{itemize}
  \tightlist
  \item
    TeXLive を入れなくても動かせるようにしたい
  \item
    たぶん \texttt{tinytex} がネック
  \end{itemize}
\item
  細かいレイアウト修正

  \begin{itemize}
  \tightlist
  \item
    例: キャプションが上か下かで統一されてない
  \end{itemize}
\item
  他の言語のシンタックスハイライト
\item
  \texttt{ggplot2} 以外で描かれたグラフの対応

  \begin{itemize}
  \tightlist
  \item
    埋め込みはできるがフォントの調整が困難
  \item
    \texttt{igraph} みたいなのとか\ldots{}
  \end{itemize}
\item
  \href{https://github.com/Gedevan-Aleksizde/my_latex_templates/labels/enhancement}{issues}
  に詳細
\end{itemize}

\end{frame}

\hypertarget{ux7d30ux304bux3044ux6280ux8853ux7684ux306aux8a71}{%
\section{細かい技術的な話}\label{ux7d30ux304bux3044ux6280ux8853ux7684ux306aux8a71}}

\begin{frame}{このセクションの想定読者}
\protect\hypertarget{ux3053ux306eux30bbux30afux30b7ux30e7ux30f3ux306eux60f3ux5b9aux8aadux8005}{}

\begin{itemize}
\tightlist
\item
  単に使いたいだけの人は見る必要なし

  \begin{itemize}
  \tightlist
  \item
    内部処理知りたい人向け
  \end{itemize}
\end{itemize}

\end{frame}

\begin{frame}[fragile]{yamlヘッダ設定: 出力の設定}
\protect\hypertarget{yamlux30d8ux30c3ux30c0ux8a2dux5b9a-ux51faux529bux306eux8a2dux5b9a}{}

\begin{itemize}
\tightlist
\item
  \XeLaTeX 生成

  \begin{itemize}
  \tightlist
  \item
    \LuaLaTeX 使用者が多数派?
  \end{itemize}
\item
  ``\texttt{keep\_tex:\ true}'' エラー発生時の原因特定に
\end{itemize}

\begin{Shaded}
\begin{Highlighting}[]
\FunctionTok{output}\KeywordTok{:}
\AttributeTok{  }\FunctionTok{beamer_presentation}\KeywordTok{:}
\AttributeTok{    }\FunctionTok{latex_engine}\KeywordTok{:}\AttributeTok{ xelatex}
\AttributeTok{    }\FunctionTok{citation_package}\KeywordTok{:}\AttributeTok{ natbib}
\AttributeTok{    }\FunctionTok{keep_tex}\KeywordTok{:}\AttributeTok{ }\CharTok{true}
\end{Highlighting}
\end{Shaded}

\end{frame}

\begin{frame}[fragile]{\LaTeX プリアンブル: テーマ設定}
\protect\hypertarget{ux30d7ux30eaux30a2ux30f3ux30d6ux30eb-ux30c6ux30fcux30deux8a2dux5b9a}{}

\begin{itemize}
\tightlist
\item
  metropolisテーマを使用

  \begin{itemize}
  \tightlist
  \item
    \url{https://github.com/matze/mtheme}
  \item
    他のモダンなテーマは\emph{日本語と相性悪い}
  \item
    ``\texttt{beamer\_presentation:}''
    内で指定すると\textbf{オプション指定できない}
  \end{itemize}
\end{itemize}

\begin{Shaded}
\begin{Highlighting}[]
\FunctionTok{header-includes}\KeywordTok{:}
\AttributeTok{  }\KeywordTok{-}\AttributeTok{ \textbackslash{}usetheme[progressbar=frametitle,block=fill]\{metropolis\}}
\AttributeTok{  }\KeywordTok{-}\AttributeTok{ \textbackslash{}makeatletter}
\AttributeTok{  }\KeywordTok{-}\AttributeTok{ \textbackslash{}setlength\{\textbackslash{}metropolis@progressinheadfoot@linewidth\}\{2pt\}}
\AttributeTok{  }\KeywordTok{-}\AttributeTok{ \textbackslash{}usefonttheme\{professionalfonts\}}
\end{Highlighting}
\end{Shaded}

\end{frame}

\begin{frame}[fragile]{\LaTeX プリアンブル: 日本語フォント設定}
\protect\hypertarget{ux30d7ux30eaux30a2ux30f3ux30d6ux30eb-ux65e5ux672cux8a9eux30d5ux30a9ux30f3ux30c8ux8a2dux5b9a}{}

\begin{itemize}
\tightlist
\item
  \texttt{zxjatype} で日本語フォント読み込み\textbf{たかった}

  \begin{itemize}
  \tightlist
  \item
    最新版で競合があるので \texttt{XeCJK} を使う
  \end{itemize}
\end{itemize}

\end{frame}

\begin{frame}{\LaTeX プリアンブル: その他の設定}
\protect\hypertarget{ux30d7ux30eaux30a2ux30f3ux30d6ux30eb-ux305dux306eux4ed6ux306eux8a2dux5b9a}{}

\begin{itemize}
\tightlist
\item
  ハイパーリンクの色を見やすく変更
\item
  ``Figure 1'', ``Table 1'' を 「図1」「表1」に
\item
  参考文献リストのフォントサイズ縮小
\item
  コードチャンクに行番号

  \begin{itemize}
  \tightlist
  \item
    表示は選択式
  \end{itemize}
\item
  その他いろいろな微調整をtexのプリアンブルで設定
\end{itemize}

\end{frame}

\begin{frame}[fragile]{日本語文献にどう対応しているか}
\protect\hypertarget{ux65e5ux672cux8a9eux6587ux732eux306bux3069ux3046ux5bfeux5fdcux3057ux3066ux3044ux308bux304b}{}

\begin{itemize}
\tightlist
\item
  \href{https://github.com/ShiroTakeda/jecon-bst/blob/master/jecon.bst}{\texttt{jecon.bst}}を使いたい

  \begin{itemize}
  \tightlist
  \item
    マルチバイト文字未対応 の\BibTeX 
  \item
    日本語は \upBibTeX 必要
  \item
    \texttt{biblatex} ではフォーマットに不満
  \end{itemize}
\item
  \texttt{knitr}は日本語書誌情報処理未対応

  \begin{itemize}
  \tightlist
  \item
    内部では自前の設定で\texttt{latexmk}を呼び出し
  \item
    呼び出しているラッパにオプションなし
  \item
    積極的に改修の気配なし(\href{https://github.com/yihui/tinytex/issues/70}{参考})
  \end{itemize}
\item
  自前の設定を使用する(\href{https://github.com/kenjimyzk/bookdown_ja_template}{参考})

  \begin{itemize}
  \tightlist
  \item
    \texttt{tinytex.latexmk.emulation\ =\ F}
  \item
    \href{https://texwiki.texjp.org/?Latexmk}{ここ}を参考に\texttt{.latexmkrc}設定
  \item
    \textbf{Rmdと同じディレクトリに}上記を置く
  \end{itemize}
\end{itemize}

\end{frame}

\begin{frame}{謝辞}
\protect\hypertarget{ux8b1dux8f9e}{}

これを作るにあたって大いに参考になった資料

\begin{itemize}
\tightlist
\item
  kazutan:
  『\href{https://kazutan.github.io/HijiyamaR6/intoTheRmarkdown.html}{R
  Markdownの内部とテンプレート開発}』
\item
  atusy:『\href{https://atusy.github.io/tokyor85-original-rmd-format/\#/}{R
  Markdownのオリジナルフォーマットを作ろう}』
\end{itemize}

\end{frame}

\begin{frame}

\section*{参考文献}

\end{frame}

\begin{frame}[allowframebreaks]{}
  \bibliographytrue
  \bibliography{examples.bib}
\end{frame}

\end{document}

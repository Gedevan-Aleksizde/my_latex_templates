

%\numberwithin{equation}{section}

%-----------------

\usepackage{ifthen}
\usepackage{booktabs}
\usepackage{longtable}
\usepackage[bf,singlelinecheck=off]{caption}
\usepackage{fontspec}
%\usepackage[AutoFallBack=true]{zxjatype} %和文フォント指定用
\usepackage{zxjatype} %和文フォント指定用
\usepackage{pxrubrica}  %ルビ


% XeLaTeX用和文フォント指定
% 欧文フォント指定はindex.Rmdでも指定できる.
\setjamainfont{Noto Serif CJK JP} %和文フォント指定
\setjasansfont{Noto Sans CJK JP} %和文サンセリフ指定
\setjamonofont{Ricty Diminished} %和文等幅フォント

% 孫ワン対策. CJK拡張漢字がいらないなら不要.
\setjafamilyfont{HMA}{花園明朝A}
\setjafamilyfont{HMB}{花園明朝B}

\usepackage{framed,color}
\definecolor{shadecolor}{RGB}{248,248,248}

\renewcommand{\textfraction}{0.05}
\renewcommand{\topfraction}{0.8}
\renewcommand{\bottomfraction}{0.8}
\renewcommand{\floatpagefraction}{0.75}

\renewenvironment{quote}{\begin{VF}}{\end{VF}}
\let\oldhref\href

% ---- ハイパーリンクを脚注に置き換えるかどうか ---
% 印刷物として出すなら必要かもしれないが
% この単純なマクロは脚注内のリンクなどを想定していない

%\renewcommand{\href}[2]{#2\footnote{\url{#1}}}


% ---- XeLaTeX 専用のあれ ----
\ifxetex
  \usepackage{letltxmacro}
  \setlength{\XeTeXLinkMargin}{1pt}
  \LetLtxMacro\SavedIncludeGraphics\includegraphics
  \def\includegraphics#1#{% #1 catches optional stuff (star/opt. arg.)
    \IncludeGraphicsAux{#1}%
  }%
  \newcommand*{\IncludeGraphicsAux}[2]{%
    \XeTeXLinkBox{%
      \SavedIncludeGraphics#1{#2}%
    }%
  }%
\fi

\makeatletter
\newenvironment{kframe}{%
\medskip{}
\setlength{\fboxsep}{.8em}
 \def\at@end@of@kframe{}%
 \ifinner\ifhmode%
  \def\at@end@of@kframe{\end{minipage}}%
  \begin{minipage}{\columnwidth}%
 \fi\fi%
 \def\FrameCommand##1{\hskip\@totalleftmargin \hskip-\fboxsep
 \colorbox{shadecolor}{##1}\hskip-\fboxsep
     % There is no \\@totalrightmargin, so:
     \hskip-\linewidth \hskip-\@totalleftmargin \hskip\columnwidth}%
 \MakeFramed {\advance\hsize-\width
   \@totalleftmargin\z@ \linewidth\hsize
   \@setminipage}}%
 {\par\unskip\endMakeFramed%
 \at@end@of@kframe}
\makeatother


% --- 特殊ブロックをLaTeXで表現するためのマクロ ---
\makeatletter
\@ifundefined{Shaded}{
}{\renewenvironment{Shaded}{\begin{kframe}}{\end{kframe}}}
\makeatother

\newenvironment{rmdblock}[1]
  {
  \begin{itemize}
  \renewcommand{\labelitemi}{
    \raisebox{-.7\height}[0pt][0pt]{
      {\setkeys{Gin}{width=3em,keepaspectratio}\includegraphics{images/#1}}
    }
  }
  \setlength{\fboxsep}{1em}
  \begin{kframe}
  \item
  }
  {
  \end{kframe}
  \end{itemize}
  }
\newenvironment{rmdnote}
  {\begin{rmdblock}{note}}
  {\end{rmdblock}}
\newenvironment{rmdcaution}
  {\begin{rmdblock}{caution}}
  {\end{rmdblock}}
\newenvironment{rmdimportant}
  {\begin{rmdblock}{important}}
  {\end{rmdblock}}
\newenvironment{rmdtip}
  {\begin{rmdblock}{tip}}
  {\end{rmdblock}}
\newenvironment{rmdwarning}
  {\begin{rmdblock}{warning}}
  {\end{rmdblock}}


% ---- 索引 ----
% TODO: 索引使わないことを想定していないぞんざいなコーディング
\usepackage{makeidx}
%\makeindex

\urlstyle{tt}

\usepackage{amsthm}
\makeatletter
\def\thm@space@setup{%
  \thm@preskip=8pt plus 2pt minus 4pt
  \thm@postskip=\thm@preskip
}
\makeatother

\frontmatter

% 環境名を日本語に
% ほとんどbookdown側で指定できるので不要
%\@ifpackageloaded{prettyref}{
%  \newrefformat{eq}{\textup{(\ref{#1})}}%
%  \newrefformat{lem}{補題\ref{#1}}%
%  \newrefformat{thm}{定理\ref{#1}}%
%  \newrefformat{prop}{命題\ref{#1}}%
%  \newrefformat{cha}{第\ref{#1}章}%
%  \newrefformat{sec}{第\ref{#1}節}%
%  \newrefformat{tab}{表\ref{#1}}%
%  \newrefformat{fig}{図\ref{#1}}%
%  \newrefformat{alg}{アルゴリズム \ref{#1}}%
%  \newrefformat{tabp}{\pageref{#1} ページの表\ref{#1}}%
%  \newrefformat{figb}{\pageref{#1} ページの図 \ref{#1}}%
%}{}
